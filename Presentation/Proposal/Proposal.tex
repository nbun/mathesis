% !TEX TS-program = xelatex
\documentclass{beamer}
\usepackage{ulem}
\usepackage{proof}
\usepackage{bussproofs}
\usepackage{amsmath}
\usepackage{amssymb}
\usepackage{minted}
\usepackage{subcaption}
\usepackage{graphicx}
\usepackage{scrextend}
\usepackage{appendixnumberbeamer}

\usemintedstyle[haskell]{trac, fontsize=\small}
\usemintedstyle[coq]{fontsize=\footnotesize}
\usemintedstyle[text]{fontsize=\footnotesize}
\newcommand{\haskellinline}[1]{\mintinline{haskell}{#1}}
\newcommand{\coqinline}[1]{\mintinline{coq}{#1}}

\usetheme[progressbar=foot]{metropolis}

\usepackage[german]{babel}
\usepackage[utf8]{inputenc}

%\setbeamersize{text margin left=1.5em,text margin right=1.5em}

\title{Modellierung von Call-Time Choice als Effekt unter Verwendung von Freien Monaden}
\subject{Modellierung von Call-Time Choice als Effekt unter Verwendung von Freien Monaden}
\date{19. Dezember 2018}
\author{Niels Bunkenburg}
\institute{ 
	Arbeitsgruppe für Programmiersprachen und Übersetzerkonstruktion \par
	Institut für Informatik \par
	Christian-Albrechts-Universität zu Kiel}

\begin{document}

\begin{frame}
  \titlepage
\end{frame}

\begin{frame}{Motivation}
\haskellinline{let x = 0?1 in x + x} $\quad \neq \quad$ \haskellinline{(0?1) + (0?1)}

\begin{itemize}
\item Ersetzungsregeln sind in Curry nicht immer wie in Haskell anwendbar
\item Beweise über die Semantik eines Programms sind schwierig
\item Ansatz: Übersetzung des Programms in andere Sprache
\item Modellierung der Effekte (z.B. Nichtdeterminismus)
\end{itemize}
\end{frame}

\begin{frame}[fragile]{Grundlagen}
\begin{itemize}
\item Programme
\begin{minted}{haskell}
data Prog sig a = Return a 
                | Op (sig (Prog sig a))
\end{minted}
\item Programmsignaturen
\begin{minted}{haskell}
data (sig1 + sig2) p = Inl (sig1 p) 
                     | Inr (sig2 p)
\end{minted}
\end{itemize}
\end{frame}

\begin{frame}[fragile]{Beispieleffekte}
\begin{itemize}
\item Effektfreie Programme
\begin{minted}{haskell}
data Void p
data VoidProg a = Return a
\end{minted}

\item Nichtdeterministische Programme
\begin{minted}{haskell}
data ND p = Fail | Choice p p
data NDProg a = Return a 
              | Fail
              | Choice (NDProg a) (NDProg a)
\end{minted}
\end{itemize}
$\rightarrow$ Der konkrete Datentyp bestimmt die \alert{Syntax} eines Effekts
\end{frame}

\begin{frame}[fragile]{Handler}
Effekte werden durch \alert{Handler} verarbeitet
\begin{minted}{haskell}
run :: Prog Void a -> a
run (Return x) = x

runND :: Prog ND a -> Tree a
runND (Return a) = Leaf a
runND Fail       = Empty
runND (Choice p q) =
  let pt = runND p
      qt = runND q
  in Branch pt qt
\end{minted}
$\rightarrow$ Der Handler bestimmt die \alert{Semantik} eines Effekts
\end{frame}

\begin{frame}[fragile]{Funktionen}
\begin{minted}{haskell}
coin :: Prog ND Int
coin = Choice (return 0) (return 1)
\end{minted}
\pause
\begin{minted}{haskell}
addM :: Prog sig Int -> Prog sig Int -> Prog sig Int
addM p1 p2 = do
  i1 <- p1
  i2 <- p2
  return $ i1 + i2
\end{minted}
$\rightarrow$ \haskellinline{liftM} bei strikten Funktionen
\pause
\begin{minted}{haskell}
orM :: Prog sig Bool -> Prog sig Bool -> Prog sig Bool
orM p1 p2 = p1 >>= \b -> case b of
                           True  -> return True
                           False -> p2
\end{minted}
$\rightarrow$ Pattern Matching erfordert Bind
\end{frame}

\begin{frame}[fragile]{Beispielausdrücke}
\begin{minted}{haskell}
λ> run $ addM (return 42) undefined
*** Exception: Prelude.undefined

λ> run $ orM (return True) undefined
True

λ> runND coin
Choice (Leaf 0) (Leaf 1)
\end{minted}
\pause
\begin{minted}{haskell}
λ> putStrLn . pretty . runND $ addM coin coin
?
|---- ?
      |---- 0
      |---- 1
|---- ?
      |---- 1
      |---- 2
\end{minted}
\end{frame}

\begin{frame}[fragile]{Call-Time Choice}
\begin{itemize}
\item Nichtdeterminismus und \alert{Sharing}
\begin{minted}{haskell}
Prelude> let x = 0 ? 1 in x + x
0
2
\end{minted}
\item \haskellinline{let} entspricht Sharing-Effekt
\begin{minted}{haskell}
?1
|---- ?1
      |---- 0
      |---- 1
|---- ?1
      |---- 1
      |---- 2
\end{minted}
\end{itemize}
$\rightarrow$ Sharing-Effekt vergibt \alert{IDs} für Choices
\end{frame}

\begin{frame}[fragile]{Sharing-Effekt -- Idee}
\begin{itemize}
\item Explizite Sharing Syntax 
\begin{minted}{haskell}
share :: Prog sig a -> Prog sig (Prog sig a)
\end{minted}
\item \haskellinline{let x = coin in x + x} \\
      wird zu\\
      \haskellinline{share coin >>= \fx -> addM fx fx}
\end{itemize}
Wie werden Choice IDs innerhalb von \haskellinline{share} vergeben?
\end{frame}

\begin{frame}[fragile]{Sharing-Effekt -- Datentyp und Handler}
\begin{minted}{haskell}
data Share p = Share p

runShare :: Prog (Share + ND) a -> Prog ND a
runShare (Return a) = return a
runShare (Share  p) = nameChoices p
runShare (Other op) = Op (fmap runShare op)

nameChoices :: Prog (Share + ND) a -> Prog ND a
nameChoices (Return a)     = return a
nameChoices (Share  p)     = nameChoices p
nameChoices Fail           = fail
nameChoices (Choice _ p q) =
  choice 42 (nameChoices p) (nameChoices q)
\end{minted}
$\rightarrow$ \haskellinline{Share} benötigt eine ID
\end{frame}

\begin{frame}[fragile]{Sharing-Effekt -- IDs}
\begin{minted}{haskell}
data Share p = Share Int p
\end{minted}
$\rightarrow$ \haskellinline{share} muss ID selbst generieren

Idee: Individuelle IDs durch \alert{State Effekt}
% Monadengesetz Linksidentität: Left identity: return a >>= f ≡ f a
\begin{minted}{haskell}
share p = return $ do
  i <- get
  put (i + 1)
  share' i p -- Smartkonstruktor für Share
\end{minted}
\end{frame}

\begin{frame}[fragile]{Sharing-Effekt -- Problem}
\begin{minted}{haskell}
do fx <- share coin
   addM fx fx
\end{minted}

\begin{minted}{haskell}
do fx <- return $ do
     i <- get
     put (i + 1)
     share' i coin
   addM fx fx -- State Code wird dupliziert!
\end{minted}
\pause
\begin{minted}{haskell}
do fx <- do -- State Code wird ausgewertet
     i <- get
     put (i + 1)
     return (share' i coin)
  addM fx fx
\end{minted}
$\rightarrow$ Zwei Programmebenen sind notwendig
\end{frame}

\begin{frame}[fragile]{Sharing-Effekt -- Nested Sharing}
\alert{Nested Sharing}

\begin{minipage}[t]{0.48\linewidth}
\begin{minted}{haskell}
let x = let y = coin
        in y + y
in x + x
\end{minted}
\end{minipage}
\begin{minipage}[t]{0.48\linewidth}
\begin{minted}{haskell}
share p = do
  i <- get
  put (i * 2)
  return . share' i $ do
    put (i * 2 + 1)
    p
\end{minted}
\end{minipage}
\vfill
$\rightarrow$ Verschachtelte Aufrufe von \haskellinline{share} benötigen frische IDs
\end{frame}

\begin{frame}[fragile]{Sharing-Effekt -- Deep Sharing}
\alert{Deep Sharing}

Geliftete Datentypen erlauben Effekte in den Komponenten, z.B.
  \haskellinline{data List m a = Nil | Cons (m a) (m (List m a))}

\vfill
\begin{minipage}[t]{0.48\linewidth}
\begin{minted}{haskell}
let xs = [coin]
in (xs, xs)
\end{minted}
\end{minipage}
\begin{minipage}[t]{0.48\linewidth}
\begin{minted}{haskell}
share p = do
  i <- get
  put (i * 2)
  return . share' i $ do
    put (i * 2 + 1)
    x <- p
    shareArgs share x
\end{minted}
\end{minipage}

$\rightarrow$ Rekursive \haskellinline{share} Aufrufe für Komponenten notwendig
\end{frame}

\begin{frame}[fragile]{Zusammenfassung und Ausblick}
Zusammenfassung
\begin{itemize}
\item Effekte können als Instanzen des Datentyps \haskellinline{Prog} modelliert werden
\item Handler setzen Effekte im Programm um
\item Call-Time Choice in Curry wird durch Nichtdeterminismus und Sharing modelliert
\item Sharing kennzeichnet Choices geschickt mit IDs
\end{itemize}
\vfill
Ausblick
\begin{itemize}
\item Drei Ansätze $\rightarrow$ Implementierung in Coq
\item Beweisen der \textit{laws of sharing} für die Implementierung
\item Beweise über Eigenschaften von konkreten Curry Programmen
\end{itemize}
\end{frame}

\appendix

\begin{frame}[fragile]{Verschachteltes Sharing}

\begin{minted}{haskell}
share (share coin >>= \fx -> addM fx fx)
  >>= \fy -> addM fy fy
\end{minted}

\begin{minipage}[t]{0.48\linewidth}
\begin{minted}{haskell}
?2
|-- ?2
    |-- ?4
        |-- ?4
            |-- 0
            |-- 1
        |-- ?4
            |-- 1
            |-- 2
            ...
\end{minted}
{\tiny Ohne Berücksichtigung von verschachteltem Sharing}
\end{minipage}
\begin{minipage}[t]{0.48\linewidth}
\begin{minted}{haskell}
?3
|-- ?3
    |-- ?3
        |-- ?3
            |-- 0
            |-- 1
        |-- ?3
            |-- 1
            |-- 2
            ...
\end{minted}
{\tiny Mit Berücksichtigung von verschachteltem Sharing}
\end{minipage}
\end{frame}

\begin{frame}[fragile]{Sharing-Effekt -- Vollständiger Handler}
\begin{minted}{haskell}
runShare :: (Functor sig, ND <: sig) 
  => Prog (Share + sig) a -> (Prog sig a)
runShare (Return a)  = return a
runShare (Share i p) = nameChoices i 1 p
runShare (Other op)  = Op (fmap runShare op)

nameChoices :: (ND <: sig) 
  => Int -> Int -> Prog (Share + sig) a -> Prog sig a
nameChoices scope next prog = case prog of
  Return a  -> Return a
  Share i p -> nameChoices i 1 p
  Choice _ p q ->
    let p' = nameChoices scope (2 * next)      p
        q' = nameChoices scope (2 * next + 1)  q
    in choiceID (Just (scope, next)) p' q'
  Other op -> Op (fmap (nameChoices scope next) op)
\end{minted}
\end{frame}

\begin{frame}{Laws of Sharing}
Purely Functional Lazy Non-deterministic Programming
\includegraphics[width=\linewidth]{img/LawsOfSharing.png}
\end{frame}

\begin{frame}[fragile]{Bäume}
\begin{minted}{haskell}
data Decision = L | R
type Memo = Map.Map ID Decision

dfs :: Memo -> Tree a -> [a]
dfs mem Failed = []
dfs mem (Leaf x) = [x]
dfs mem (Choice Nothing t1 t2) = dfs mem t1 
                              ++ dfs mem t2
dfs mem (Choice (Just n) t1 t2) =
    case Map.lookup n mem of
      Nothing -> dfs (Map.insert n L mem) t1 
              ++ dfs (Map.insert n R mem) t2
      Just L -> dfs mem t1
      Just R -> dfs mem t2
\end{minted}
\end{frame}

\begin{frame}[fragile]{Explizite Scopes}
\begin{minted}{haskell}
data Share p = BShare Int p | EShare Int p

share p = do
  i <- get
  put (i * 2)
  return $ do
    inject (BShare i (return ()))
    put (i * 2 + 1)
    x <- p
    x' <- shareArgs share x
    inject (EShare i (return ()))
    return x'
\end{minted}
\end{frame}
\end{document}


