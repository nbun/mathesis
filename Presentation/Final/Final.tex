% !TEX TS-program = xelatex
\documentclass{beamer}
\usepackage{ulem}
\usepackage{proof}
\usepackage{bussproofs}
\usepackage{amsmath}
\usepackage{amssymb}
\usepackage{minted}
\usepackage{subcaption}
\usepackage{graphicx}
\usepackage{scrextend}
\usepackage{appendixnumberbeamer}
\usepackage{marvosym}

\usemintedstyle[haskell]{trac, fontsize=\small}
\usemintedstyle[coq]{fontsize=\footnotesize}
\usemintedstyle[text]{fontsize=\footnotesize}
\newcommand{\haskellinline}[1]{\mintinline{haskell}{#1}}
\newcommand{\coqinline}[1]{\mintinline{coq}{#1}}

\usetheme[progressbar=foot]{metropolis}
%\setbeamertemplate{footline}[frame number]{}

\definecolor{Purple}{HTML}{253340}
\definecolor{Orange}{HTML}{cb5031}

\setbeamercolor{alerted text}{fg=Orange}
\setbeamercolor{frametitle}{bg=Purple}

\usepackage[german]{babel}
\usepackage[utf8]{inputenc}

%\setbeamersize{text margin left=1.5em,text margin right=1.5em}

\title{Modellierung von Call-Time Choice als Effekt unter Verwendung von Freien Monaden}
\subject{Modellierung von Call-Time Choice als Effekt unter Verwendung von Freien Monaden}
\date{27. März 2018}
\author{Niels Bunkenburg}
\institute{ 
	Arbeitsgruppe für Programmiersprachen und Übersetzerkonstruktion \par
	Institut für Informatik \par
	Christian-Albrechts-Universität zu Kiel}

\begin{document}

\begin{frame}
  \titlepage
\end{frame}

\begin{frame}{Motivation}
\haskellinline{let x = 0 ? 1 in x + x} $\quad \neq \quad$ \haskellinline{(0 ? 1) + (0 ? 1)}

\begin{itemize}
\item Ersetzungsregeln sind in Curry nicht immer wie in Haskell anwendbar
\item Beweise über die Semantik eines Programms sind schwierig
\item Ansatz: Übersetzung des Programms in andere Sprache
\item Modellierung der Effekte (z.B. Nichtdeterminismus)
\end{itemize}
\end{frame}

\begin{frame}[fragile]{Programm und Effektsyntax}
\begin{itemize}
\item Programm
\begin{minted}{haskell}
data Prog sig a  = Return a 
                 | Op (sig (Prog sig a))
\end{minted}

\item Effektsyntax
\begin{minted}{haskell}
data ND p = Fail | Choice p p
\end{minted}

\vspace{1em}
\hspace{7.8em}
\scalebox{3}[3]{\MVArrowDown}
\vspace{0.5em}

\item Nichtdeterministisches Programm \mintinline{haskell}{Prog ND} entspricht
\begin{minted}{haskell}
data NDProg a = Return a 
              | Fail
              | Choice (NDProg a) (NDProg a)
\end{minted}
\end{itemize}
\end{frame}

\begin{frame}[fragile]{Handler}
Effekte werden durch \alert{Handler} verarbeitet
\begin{minted}{haskell}
runND :: Prog ND a -> [a]
runND (Return a)   = [a]
runND Fail         = []
runND (Choice p q) = runND p ++ runND q
\end{minted}
$\rightarrow$ Der Handler bestimmt die \alert{Semantik} eines Effekts
\end{frame}

\begin{frame}[fragile]{Funktionen}
\begin{minted}{haskell}
coin :: Prog ND Int
coin = Choice (return 0) (return 1)
\end{minted}
\pause
\begin{minted}{haskell}
addM :: Prog sig Int -> Prog sig Int -> Prog sig Int
addM p1 p2 = do
  i1 <- p1
  i2 <- p2
  return $ i1 + i2
\end{minted}
$\rightarrow$ \haskellinline{liftM} bei strikten Funktionen
\pause
\begin{minted}{haskell}
orM :: Prog sig Bool -> Prog sig Bool -> Prog sig Bool
orM p1 p2 = p1 >>= \b -> case b of
                           True  -> return True
                           False -> p2
\end{minted}
$\rightarrow$ Pattern Matching erfordert Bind
\end{frame}

\begin{frame}[fragile]{Beispielausdrücke}
\begin{minted}{haskell}
λ> run $ addM (return 42) undefined
*** Exception: Prelude.undefined

λ> run $ orM (return True) undefined
True

λ> runND coin
[0,1]
\end{minted}
\pause
\begin{minted}{haskell}
λ> putStrLn . pretty . runND $ addM coin coin
?
|---- ?
      |---- 0
      |---- 1
|---- ?
      |---- 1
      |---- 2
\end{minted}
\end{frame}

\begin{frame}[fragile]{Call-Time Choice}
\begin{itemize}
\item Nichtdeterminismus und \alert{Sharing}
\begin{minted}{haskell}
Prelude> let x = 0 ? 1 in x + x
0
2
\end{minted}
\item \haskellinline{let} entspricht Sharing-Effekt
\begin{minted}[escapeinside=<>]{haskell}
?<$_1$>
|---- ?<$_1$>
      |---- 0
      |---- 1
|---- ?<$_1$>
      |---- 1
      |---- 2
\end{minted}
\end{itemize}
$\rightarrow$ Sharing-Effekt vergibt \alert{IDs} für Choices
\end{frame}


\begin{frame}[fragile]{Sharing als Effekt mit Scope}
\begin{minted}{haskell}
data Share p = Share p
\end{minted}

\pause

\tikz[remember picture,overlay] \node[opacity=0.75,inner sep=0pt] at 
	(4,0.93){\includegraphics[width=3em]{img/crossout.png}};\\
Gegenbeispiel

\begin{minted}{haskell}
let x = coin in (x + coin) + (x + coin)
\end{minted}
\pause
\vspace*{1em}
\begin{minipage}{.47 \linewidth}
\begin{minted}[escapeinside=||, fontsize=\footnotesize]{text}
< ?|$_\texttt{1}$|
  ├── ?|$_\texttt{\textcolor{red}{2}}$|
  │   ├── < ?|$_\texttt{1}$|
  │   │     ├── ?|$_\texttt{\textcolor{red}{2}}$|
  │   │     │   ├── 0 > >
  .   .     .   └── 1 > >
\end{minted}
\begin{center}
Einfache Implementierung
\end{center}
\end{minipage}
\vline
\hspace{.5em}
\begin{minipage}{.475 \linewidth}
\begin{minted}[escapeinside=||, fontsize=\footnotesize]{text}
< ?|$_\texttt{1}$|
  ├── > ? 
  │     ├── < ?|$_\texttt{1}$|
  │     │     ├── > ? 
  │     │     │     ├── 0
  .     .     .     └── 1
\end{minted}
\begin{center}
Richtige Implementierung
\end{center}
\end{minipage}
\end{frame}

\begin{frame}[fragile]{Sharing-Effekt -- Korrekte Implementierung}
\begin{minted}{haskell}
data Share p = BShare Int p | EShare Int p
\end{minted}
\end{frame}


\begin{frame}[fragile]{Sharing-Effekt -- Problem}
\begin{minted}{haskell}
do fx <- share coin
   addM fx fx
\end{minted}

\begin{minted}{haskell}
do fx <- return $ do
     i <- get
     put (i + 1)
     share' i coin
   addM fx fx -- State Code wird dupliziert!
\end{minted}
\pause
\begin{minted}{haskell}
do fx <- do -- State Code wird ausgewertet
     i <- get
     put (i + 1)
     return (share' i coin)
  addM fx fx
\end{minted}
$\rightarrow$ Zwei Programmebenen sind notwendig
\end{frame}

\begin{frame}[fragile]{Sharing-Effekt -- Nested Sharing}
\alert{Nested Sharing}

\begin{minipage}[t]{0.48\linewidth}
\begin{minted}{haskell}
let x = let y = coin
        in y + y
in x + x
\end{minted}
\end{minipage}
\begin{minipage}[t]{0.48\linewidth}
\begin{minted}{haskell}
share p = do
  i <- get
  put (i * 2)
  return . share' i $ do
    put (i * 2 + 1)
    p
\end{minted}
\end{minipage}
\vfill
$\rightarrow$ Verschachtelte Aufrufe von \haskellinline{share} benötigen frische IDs
\end{frame}

\begin{frame}[fragile]{Sharing-Effekt -- Deep Sharing}
\alert{Deep Sharing}

Geliftete Datentypen erlauben Effekte in den Komponenten, z.B.
  \haskellinline{data List m a = Nil | Cons (m a) (m (List m a))}

\vfill
\begin{minipage}[t]{0.48\linewidth}
\begin{minted}{haskell}
let xs = [coin]
in (xs, xs)
\end{minted}
\end{minipage}
\begin{minipage}[t]{0.48\linewidth}
\begin{minted}{haskell}
share p = do
  i <- get
  put (i * 2)
  return . share' i $ do
    put (i * 2 + 1)
    x <- p
    shareArgs share x
\end{minted}
\end{minipage}

$\rightarrow$ Rekursive \haskellinline{share} Aufrufe für Komponenten notwendig
\end{frame}

\begin{frame}[fragile]{Zusammenfassung und Ausblick}
Zusammenfassung
\begin{itemize}
\item Effekte können als Instanzen des Datentyps \haskellinline{Prog} modelliert werden
\item Handler setzen Effekte im Programm um
\item Call-Time Choice in Curry wird durch Nichtdeterminismus und Sharing modelliert
\item Sharing kennzeichnet Choices geschickt mit IDs
\end{itemize}
\vfill
Ausblick
\begin{itemize}
\item Drei Ansätze $\rightarrow$ Implementierung in Coq
\item Beweisen der \textit{laws of sharing} für die Implementierung
\item Beweise über Eigenschaften von konkreten Curry Programmen
\end{itemize}
\end{frame}

\appendix
\end{document}


