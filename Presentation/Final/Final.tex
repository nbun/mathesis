% !TEX TS-program = xelatex
\documentclass{beamer}
\usepackage{ulem}
\usepackage{proof}
\usepackage{bussproofs}
\usepackage{amsmath}
\usepackage{amssymb}
\usepackage{minted}
\usepackage{subcaption}
\usepackage{graphicx}
\usepackage{scrextend}
\usepackage{appendixnumberbeamer}
\usepackage{marvosym}

\usetheme[progressbar=foot]{metropolis}
%\setbeamertemplate{footline}[frame number]{}

\definecolor{Purple}{HTML}{253340}
\definecolor{Orange}{HTML}{cb5031}

\setbeamercolor{alerted text}{fg=Orange}
\setbeamercolor{frametitle}{bg=Purple}

\usepackage[german]{babel}
\usepackage[utf8]{inputenc}

\usemintedstyle[haskell]{trac, fontsize=\small}
\usemintedstyle[coq]{default, fontsize=\small}
\usemintedstyle[text]{default, fontsize=\small}
\newcommand{\haskellinline}[1]{\mintinline{haskell}{#1}}
\newcommand{\coqinline}[1]{\mintinline{coq}{#1}}

%\setbeamersize{text margin left=1.5em,text margin right=1.5em}

\title{Modellierung von Call-Time Choice als Effekt unter Verwendung von Freien Monaden}
\subject{Modellierung von Call-Time Choice als Effekt unter Verwendung von Freien Monaden}
\date{27. März 2019}
\author{Niels Bunkenburg}
\institute{ 
	Arbeitsgruppe für Programmiersprachen und Übersetzerkonstruktion \par
	Institut für Informatik \par
	Christian-Albrechts-Universität zu Kiel}

\begin{document}

\begin{frame}
  \titlepage
\end{frame}

\begin{frame}{Motivation}
\haskellinline{let x = 0 ? 1 in x + x} $\quad \neq \quad$ \haskellinline{(0 ? 1) + (0 ? 1)}

\begin{itemize}
\item Ersetzungsregeln sind in Curry nicht immer wie in Haskell anwendbar
\item Beweise über die Semantik eines Programms sind schwierig
\item Ansatz: Übersetzung des Programms in andere Sprache
\item Modellierung der Effekte (z.B. Nichtdeterminismus)
\end{itemize}
\end{frame}

\begin{frame}[fragile]{Programm und Effektsyntax}
\begin{itemize}
\item Programm
\begin{minted}{haskell}
data Prog sig a  = Return a 
                 | Op (sig (Prog sig a))
\end{minted}

\item Effektsyntax
\begin{minted}{haskell}
data ND p = Fail | Choice p p
\end{minted}

\vspace{1em}
\hspace{7.8em}
\scalebox{3}[3]{\MVArrowDown}
\vspace{0.5em}

\item Nichtdeterministisches Programm \mintinline{haskell}{Prog ND} entspricht
\begin{minted}{haskell}
data NDProg a = Return a 
              | Fail
              | Choice (NDProg a) (NDProg a)
\end{minted}
\item \mintinline{haskell}{Prog f} ist eine Monade, wenn \mintinline{haskell}{f} ein Funktor ist
\end{itemize}
\end{frame}

\begin{frame}[fragile]{Handler}
Effekte werden durch \alert{Handler} verarbeitet
\begin{minted}{haskell}
runND :: Prog ND a -> [a]
runND (Return a)   = [a]
runND Fail         = []
runND (Choice p q) = runND p ++ runND q
\end{minted}
\MVRightArrow{} Der Handler bestimmt die \alert{Semantik} eines Effekts
\end{frame}

\begin{frame}[fragile]{Funktionen}
\begin{minted}{haskell}
coin :: Prog ND Int
coin = Choice (return 0) (return 1)
\end{minted}
\pause
\begin{minted}{haskell}
addM :: Prog sig Int -> Prog sig Int -> Prog sig Int
addM p1 p2 = do
  i1 <- p1
  i2 <- p2
  return $ i1 + i2
\end{minted}
\MVRightArrow{} \haskellinline{liftM} bei strikten Funktionen
\pause
\begin{minted}{haskell}
orM :: Prog sig Bool -> Prog sig Bool -> Prog sig Bool
orM p1 p2 = p1 >>= \b -> case b of
                           True  -> return True
                           False -> p2
\end{minted}
\MVRightArrow{} Pattern Matching erfordert Bind
\end{frame}
\begin{frame}[fragile]{Beispielausdrücke}
\begin{minted}{haskell}
λ> run $ addM (return 42) Fail
Fail
\end{minted}
\begin{minted}{haskell}
λ> run $ orM (return True) Fail
True
\end{minted}
\begin{minted}{haskell}
λ> runND coin
[0,1]
\end{minted}
\pause
\begin{minted}{haskell}
λ> putStrLn . pretty $ addM coin coin
\end{minted}
\vspace*{-3pt}
\begin{minted}{text}
?
├── ?
│   ├── 0
│   └── 1
└── ?
    ├── 1
    └── 2
\end{minted}
\end{frame}

\begin{frame}[fragile]{Call-Time Choice}
\begin{itemize}
\item Nichtdeterminismus und \alert{Sharing}
\begin{minted}{haskell}
Prelude> let x = 0 ? 1 in x + x
0
2
\end{minted}
\item \haskellinline{let} entspricht Sharing-Effekt
\begin{minted}[escapeinside=<>]{text}
?<$_1$>
├── ?<$_1$>
│   ├── 0
│   └── 1
└── ?<$_1$>
    ├── 1
    └── 2
\end{minted}
\end{itemize}
\MVRightArrow{} Sharing-Effekt vergibt \alert{IDs} für Choices
\end{frame}

\begin{frame}[fragile]{Sharing als Effekt mit Scope}
\begin{minted}{haskell}
data Share p = Share p
\end{minted}

\pause

\tikz[remember picture,overlay] \node[opacity=0.75,inner sep=0pt] at 
	(4,0.93){\includegraphics[width=3em]{img/crossout.png}};\\
Gegenbeispiel:

\begin{minted}{haskell}
let x = coin in (x + coin) + (x + coin)
\end{minted}
\pause
\vspace*{1em}
\begin{minipage}{.47 \linewidth}
\begin{minted}[escapeinside=||, fontsize=\footnotesize]{text}
< ?|$_\texttt{1}$|
  ├── ?|$_\texttt{\textcolor{red}{2}}$|
  │   ├── < ?|$_\texttt{1}$|
  │   │     ├── ?|$_\texttt{\textcolor{red}{2}}$|
  │   │     │   ├── 0 > >
  .   .     .   └── 1 > >
\end{minted}
\begin{center}
Einfache Implementierung
\end{center}
\end{minipage}
\vline
\hspace{.5em}
\begin{minipage}{.475 \linewidth}
\begin{minted}[escapeinside=||, fontsize=\footnotesize]{text}
< ?|$_\texttt{1}$|
  ├── > ? 
  │     ├── < ?|$_\texttt{1}$|
  │     │     ├── > ? 
  │     │     │     ├── 0
  .     .     .     └── 1
\end{minted}
\begin{center}
Richtige Implementierung
\end{center}
\end{minipage}
\end{frame}

\begin{frame}[fragile]{Sharing-Effekt -- Korrekte Implementierung}
\begin{minted}{haskell}
data Share p = BShare ID p | EShare ID p
\end{minted}

\MVRightArrow{} Smartkonstruktor notwendig, um IDs zu generieren
\pause
\begin{minted}{haskell}
share :: Prog sig a -> Prog sig (Prog sig a)
share p = do
  i <- get
  put (i + 1)
  return $ do
    begin i
    x <- p
    end i
    return x
\end{minted}
\end{frame}


\begin{frame}[fragile]{Sharing-Effekt -- Nested Sharing}
\alert{Nested Sharing}

Sharing kann beliebig tief verschachtelt auftreten
\vfill
Beispiel:
\begin{minted}{haskell}
let x = let y = coin
        in y + y
in x + x
\end{minted}

\MVRightArrow{} Verschachtelte Aufrufe von \haskellinline{share} benötigen frische IDs
\end{frame}

\begin{frame}[fragile]{Sharing-Effekt -- Deep Sharing}
\alert{Deep Sharing}

Geliftete Datentypen erlauben Effekte in den Komponenten

\begin{minted}{haskell}
data List m a = Nil | Cons (m a) (m (List m a))
\end{minted}
\vfill
Beispiel:
\begin{minted}{haskell}
let xs = [coin]
in (head xs, head xs)
\end{minted}

\MVRightArrow{} Rekursive \haskellinline{share} Aufrufe für Komponenten notwendig
\end{frame}

\begin{frame}{Coq}
\begin{itemize}
\item Interaktiver Beweisassistent mit funktionaler Spezifikationssprache
\item Alle Definition müssen nachweislich terminieren
\end{itemize}
Interessante Aspekte im Vergleich zu Haskell:
\begin{itemize}
\item Modellierung von Nicht-Striktheit in strikter Sprache
\item Beweise über Curry Programme
\end{itemize}

\tikz[remember picture,overlay] \node[opacity=0.2,inner sep=0pt] at (9.5,-1){\includegraphics{img/coq.png}};
\end{frame}

\begin{frame}[fragile]{Implementierung in Coq: Prog}
\mintinline{haskell}{Prog} entspricht der Freien Monade \mintinline{haskell}{Free}
\begin{minted}{coq}
Inductive Free F A :=
| pure : A -> Free F A
| impure : F (Free F A) -> Free F A.
\end{minted}
\hspace{.39\linewidth}
\scalebox{3}[3]{\MVArrowDown}
\begin{minted}{coq}
Non-strictly positive occurrence of "Free" in
  "F (Free F A) -> Free F A".
\end{minted}
\vfill
\MVRightArrow{} Alternative Darstellung von Funktoren möglich?
\end{frame}

\begin{frame}[fragile]{Darstellung von Funktoren als Container}
\alert{Container} abstrahieren Datentypen, die (polymorphe) Werte enthalten
\begin{minted}{coq}
Class Container :=
  {
    Shape   : Type;
    Pos     : Shape -> Type
  }.

Inductive Ext (C : Container) A := 
  ext : forall s : Shape,
          (Pos s -> A) -> Ext C A.
\end{minted}

Die \alert{Erweiterung eines Containers} ist isomorph zum ursprünglichen Datentyp
\end{frame}

\begin{frame}[fragile]{Funktoren als Container: Choice}
\begin{minted}{coq}
Inductive Choice (A : Type) :=
| cfail   : Choice A
| cchoice : option ID -> A -> A -> Choice A.

Inductive ShapeChoice :=
| sfail   : ShapeChoice
| schoice : option ID -> ShapeChoice.

Definition PosChoice (s: ShapeChoice) : Type :=
  match s with
  | sfail     => Void
  | schoice _ => bool
  end.
\end{minted}
\end{frame}

\begin{frame}[fragile]{Container Erweiterung für Choice}
\begin{minted}[fontsize=\footnotesize]{coq}     
Instance CChoice : Container := {
  Shape := ShapeChoice;
  Pos   := PosChoice
}.

Definition fromChoice A (z : Choice A) : Ext CChoice A :=
  match z with
  | cfail _  => 
      ext sfail 
          (fun p : Void => match p with end)
  | cchoice mid l r =>
      ext (schoice mid) 
          (fun p : bool => if p then l else r)
  end.
\end{minted}
\end{frame}

\begin{frame}[fragile]{Übersicht Coq Implementierung}
\begin{itemize}
\item Darstellung aller Effekte und des Kombinationsfunktors als Container
\item Statischer Effekt-Stack und Handlerreihenfolge
\item Implementierung des \texttt{begin/end} Ansatzes für Sharing
\item Problematisch: Handling von ungültigen \texttt{begin/end} Tags
\end{itemize}
\MVRightArrow{} Higher-Order Ansatz
\end{frame}

\begin{frame}[fragile]{Zusammenfassung und Ausblick}
Zusammenfassung
\begin{itemize}
\item Effekte können als Instanzen des Datentyps \haskellinline{Prog} modelliert werden
\item Handler setzen Effekte im Programm um
\item Call-Time Choice in Curry wird durch Nichtdeterminismus und Sharing modelliert
\item Coq: Darstellung von Funktoren als Container
\end{itemize}
\vfill
Ausblick
\begin{itemize}
\item Ist es möglich, den Higher-Order Ansatz vollständig in Coq zu modellieren?
\item Lassen sich andere Effekte, die von Sharing beeinflusst werden, ebenfalls darstellen?
\end{itemize}
\end{frame}

\appendix

\begin{frame}[fragile]{Higher-Order Implementierung}
\begin{minted}{haskell}
data HShare m a = forall x. Share ID (m x) (x -> m a)
\end{minted}
\begin{itemize}
\item Datentypen haben zusätzliches Argument \texttt{m}
\item \texttt{(m a)} $\hat{=}$ \texttt{p} in bisheriger Implementierung
\item Effekt Scope durch direkte Programmargumente
\item Coq Implementierung: Indizierte Bi-Container
\item Rekursive Datentypen nicht darstellbar
\end{itemize}
\end{frame}

\begin{frame}[fragile]{Sharing Operator}
\begin{minted}{haskell}
share :: Prog sig a -> Prog sig (Prog sig a)
share p = do
  (i, j) <- get
  put (i + 1, j)
  return $ do
    begin (i,j)
    put (i, j + 1)
    x <- p
    x' <- shareArgs share x
    end (i,j)
    return x'
\end{minted}
\end{frame}
\end{document}
